% fontsize.tex
% Copyright (C) 2018 Kelly Smith.
%
% This work may be distributed and/or modified under the
% conditions of the LaTeX Project Public License (LPPL),
% either version 1.3c of this license or (at your option)
% any later version. The latest version of this license
% is in the file:
%
%   http://www.latex-project.org/lppl.txt
%
% This work has the LPPL maintenance status `maintained'.
% 
% The Current Maintainer of this work is Kelly Smith.
%
% This work consists of the files listed in the README.

\ProvidesExplFile{fontsize.tex}{2018/10/15}{0.6.0}{Font size template system for the Soranus class.}


% Here, I create aliases to the TeX primitives.

\cs_new:Npn \g__soranus_top_skip
  {\tex_topskip:D}

\cs_new:Npn \g__soranus_max_depth_skip
  {\tex_maxdepth:D}

\cs_new:Npn \l__soranus_above_display_skip
  {\tex_abovedisplayskip:D}

\cs_new:Npn \l__soranus_above_display_short_skip
  {\tex_abovedisplayshortskip:D}

\cs_new:Npn \l__soranus_below_display_skip
  {\tex_belowdisplayskip:D}

\cs_new:Npn \l__soranus_below_display_short_skip
  {\tex_belowdisplayshortskip:D}


% Here I define global font size variables.
\dim_new:N \g__soranus_point_size_dim
\fp_new:N \g__soranus_baseline_scale_fp


% Utilities for scaling dimensions.
\cs_new:Nn \soranus_scale_to_nearest_nth_point:nnn
  {\fp_to_dim:n {round(#3 * #2 * \dim_to_fp:n {#1}) / #3}}

\NewDocumentCommand \ScaleToNearestPoint {m m}
  {\soranus_scale_to_nearest_nth_point:nnn {#1} {#2} {1}}

\NewDocumentCommand \ScaleToNearestHalfPoint {m m}
  {\soranus_scale_to_nearest_nth_point:nnn {#1} {#2} {2}}

\NewDocumentCommand \ScaleToNearestQuarterPoint {m m}
  {\soranus_scale_to_nearest_nth_point:nnn {#1} {#2} {4}}

\NewDocumentCommand \ScaleToNearestEighthPoint {m m}
  {\soranus_scale_to_nearest_nth_point:nnn {#1} {#2} {8}}

\NewDocumentCommand \ScaleToNearestTenthPoint {m m}
  {\soranus_scale_to_nearest_nth_point:nnn {#1} {#2} {10}}


% User commands for safely accessing document point size.
\NewDocumentCommand \NormalPointSize {}
  {\dim_use:N \g__soranus_point_size_dim}

\NewDocumentCommand \NormalBaselineScale {}
  {\fp_to_decimal:N \g__soranus_baseline_scale_fp}

% Here I define the template variables.

\dim_new:N \l__soranus_fontsize_point_size_dim
\fp_new:N \l__soranus_fontsize_baseline_scale_fp
\bool_new:N \l__soranus_fontsize_change_display_bool
\skip_new:N \l__soranus_fontsize_above_display_skip
\skip_new:N \l__soranus_fontsize_above_display_short_skip
\skip_new:N \l__soranus_fontsize_below_display_skip
\skip_new:N \l__soranus_fontsize_below_display_short_skip


% Define an object that represents a font size command.
\DeclareObjectType {fontsize} {0}


% Define the interface for the standard (only) font size template.
\DeclareTemplateInterface {fontsize} {standard} {0}
  {
    point-size               : length  = 10pt,
    baseline-scale           : real    = 1.2,
    change-display-skips     : boolean = false,
    above-display-skip       : skip    = 0pt,
    above-display-short-skip : skip    = 0pt,
    below-display-skip       : skip    = 0pt,
    below-display-short-skip : skip    = 0pt
  }


% I forbid normal font size switches in the middle of a math environment.
\msg_new:nnn {soranus} {fontsize-in-mmode}
  {
    Normal~font~size~commands~are~forbidden~in~math~mode.~Command~ignored.
  }

% 
\DeclareTemplateCode {fontsize} {standard} {0}
  {
    point-size               = \l__soranus_fontsize_point_size_dim,
    baseline-scale           = \l__soranus_fontsize_baseline_scale_fp,
    change-display-skips     = \l__soranus_fontsize_change_display_bool,
    above-display-skip       = \l__soranus_fontsize_above_display_skip,
    above-display-short-skip = \l__soranus_fontsize_above_display_short_skip,
    below-display-skip       = \l__soranus_fontsize_below_display_skip,
    below-display-short-skip = \l__soranus_fontsize_below_display_short_skip
  }
  {
    \mode_if_math:TF
      {
        \msg_warning:nn {soranus} {fontsize-in-mmode}
      }
      {
        \AssignTemplateKeys
        % I do not use \@setfontsize since I handle math mode in my own way
        % and because \@currsize is of no use in this class.
        \fontsize
          \l__soranus_fontsize_point_size_dim
          {
            \ScaleToNearestTenthPoint
              {\l__soranus_fontsize_point_size_dim}
              {\l__soranus_fontsize_baseline_scale_fp}
          }
        \selectfont
        \bool_if:nT {\l__soranus_fontsize_change_display_bool}
          {
            \skip_set_eq:NN \l__soranus_above_display_skip
              \l__soranus_fontsize_above_display_skip
            \skip_set_eq:NN \l__soranus_above_display_short_skip
              \l__soranus_fontsize_above_display_short_skip
            \skip_set_eq:NN \l__soranus_below_display_skip
              \l__soranus_fontsize_below_display_skip
            \skip_set_eq:NN \l__soranus_below_display_short_skip
              \l__soranus_fontsize_below_display_short_skip
          }
      }
  }


%%%%%%%%%%%%%%%%%%%%
%%%%%%%%%%%%%%%%%%%%
% Instances
%
% These default values set the baselineskip to 1.2 times the font size,
% rounded to the nearest quarter point. The font sizes themselves follow
% the series of scales relative to the normal size in this table:
%
%   miniscule     0.5
%   tiny          0.6
%   scriptsize    0.7
%   footnotesize  0.8
%   small         0.9
%   normalsize    1.0
%   large         1.2
%   Large         1.4
%   LARGE         1.6
%   huge          2.4
%   Huge          2.8
%   HUGE          3.2

\DeclareInstance {fontsize} {miniscule} {standard}
  {
    point-size     = \ScaleToNearestTenthPoint {\NormalPointSize} {0.5},
    baseline-scale = \NormalBaselineScale
  }

\DeclareInstance {fontsize} {tiny} {standard}
  {
    point-size     = \ScaleToNearestTenthPoint {\NormalPointSize} {0.6},
    baseline-scale = \NormalBaselineScale
  }

\DeclareInstance {fontsize} {scriptsize} {standard}
  {
    point-size     = \ScaleToNearestTenthPoint {\NormalPointSize} {0.7},
    baseline-scale = \NormalBaselineScale
  }

\DeclareInstance {fontsize} {footnotesize} {standard}
  {
    point-size               = \ScaleToNearestTenthPoint {\NormalPointSize} {0.8},
    baseline-scale           = \NormalBaselineScale,
    change-display-skips     = true,
    above-display-skip       = 6pt plus 2pt minus 4pt,
    above-display-short-skip = 0pt plus 1pt,
    below-display-skip       = \KeyValue{above-display-skip},
    below-display-short-skip = 3pt plus 1pt minus 2pt
  }

\DeclareInstance {fontsize} {small} {standard}
  {
    point-size               = \ScaleToNearestTenthPoint {\NormalPointSize} {0.9},
    baseline-scale           = \NormalBaselineScale,
    change-display-skips     = true,
    above-display-skip       = 8.5pt plus 3pt minus 4pt,
    above-display-short-skip = 0pt plus 2pt,
    below-display-skip       = \KeyValue{above-display-skip},
    below-display-short-skip = 4pt plus 2pt minus 2pt
  }

\DeclareInstance {fontsize} {normalsize} {standard}
  {
    point-size               = \NormalPointSize,
    baseline-scale           = \NormalBaselineScale,
    change-display-skips     = true,
    above-display-skip       = 10pt plus 2pt minus 5pt,
    above-display-short-skip = 0pt plus 3pt,
    below-display-skip       = \KeyValue{above-display-skip},
    below-display-short-skip = 6pt plus 3pt minus 3pt
  }

\DeclareInstance {fontsize} {large} {standard}
  {
    point-size     = \ScaleToNearestTenthPoint {\NormalPointSize} {1.2},
    baseline-scale = \NormalBaselineScale
  }

\DeclareInstance {fontsize} {Large} {standard}
  {
    point-size     = \ScaleToNearestTenthPoint {\NormalPointSize} {1.4},
    baseline-scale = \NormalBaselineScale
  }

\DeclareInstance {fontsize} {LARGE} {standard}
  {
    point-size     = \ScaleToNearestTenthPoint {\NormalPointSize} {1.6},
    baseline-scale = \NormalBaselineScale
  }

\DeclareInstance {fontsize} {huge} {standard}
  {
    point-size     = \ScaleToNearestTenthPoint {\NormalPointSize} {2.4},
    baseline-scale = \NormalBaselineScale
  }

\DeclareInstance {fontsize} {Huge} {standard}
  {
    point-size     = \ScaleToNearestTenthPoint {\NormalPointSize} {2.8},
    baseline-scale = \NormalBaselineScale
  }

\DeclareInstance {fontsize} {HUGE} {standard}
  {
    point-size     = \ScaleToNearestTenthPoint {\NormalPointSize} {3.2},
    baseline-scale = \NormalBaselineScale
  }




%%%%%%%%%%%%%%%%%%%%
%%%%%%%%%%%%%%%%%%%%
% Document Size Commands
%
% Now, a font size is changed by editing its respective instance.

\NewDocumentCommand \miniscule {}
  {\UseInstance {fontsize} {miniscule}}

\NewDocumentCommand \tiny {}
  {\UseInstance {fontsize} {tiny}}

\NewDocumentCommand \scriptsize {}
  {\UseInstance {fontsize} {scriptsize}}

\NewDocumentCommand \footnotesize {}
  {\UseInstance {fontsize} {footnotesize}}

\NewDocumentCommand \small {}
  {\UseInstance {fontsize} {small}}

\RenewDocumentCommand \normalsize {}
  {\UseInstance {fontsize} {normalsize}}

\NewDocumentCommand \large {}
  {\UseInstance {fontsize} {large}}

\NewDocumentCommand \Large {}
  {\UseInstance {fontsize} {Large}}

\NewDocumentCommand \LARGE {}
  {\UseInstance {fontsize} {LARGE}}

\NewDocumentCommand \huge {}
  {\UseInstance {fontsize} {huge}}

\NewDocumentCommand \Huge {}
  {\UseInstance {fontsize} {Huge}}

\NewDocumentCommand \HUGE {}
  {\UseInstance {fontsize} {HUGE}}
